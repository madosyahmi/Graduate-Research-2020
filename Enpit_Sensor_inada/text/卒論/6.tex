%あとがき,結論
%5-1研究の成果(考えたこと,わかったこと)を書く(結論表),
%5-2成果の価値(学術的価値や実用的価値)を書く(先発性・明晰性・普遍性・可用性),
%5-3今後の課題(成果の限界とその克服のための戦略)を書く(アイデアの平凡さ・あいまいさ・狭さ・非実用性を検討

本論文では,環境値収集による感染予防サポートシステムの開発を行った.
感染症へのリスクが高まる状況を作らないことが求められるようになってきているが,リスクを上げないために換気を行っても,どの程度環境が変わったかは分かりにくい.
換気を表す指標として二酸化炭素濃度が有用であるといわれているが,その計測機器は設置箇所への制限が多く,安易に導入しにくい.
そこで,本研究においては,感染症リスクを分かりやすく表示でき,容易に設置できる感染症予防サポートシステムを提案した.
この開発においては,V字開発モデルに従い,UML図を使用した設計を通じてグループでの開発を行った.
%V字モデル行うことで設計と検証の関係が明らかとなり,またUML図により,設計をビジュアル的に,そしてオブジェクト指向にのっとって表現することができた.
本システムによって,部屋の環境値の収集を配線不要の小さなデバイスで行うことができるようになった.
また,部屋の特性を踏まえた感染リスクを分かりやすく表示できるものとなった.

これらのシステムは,感染症を防ぐのに利用でき,また利用者の感染症への意識向上を促すものとなりうると考えられる.
それだけでなく,様々な場所の感染リスクや環境値の変動を収集できることは,人の移動の予想やよりよい感染症予防のガイドラインの制定にも応用が可能であると考えられる.
環境値の測定が容易に,多くの場所で行えるということは大きな利益を生みうるということを本システムにより確認することができた.

%本論文ではセンシング技術を用いたモビリティショッピング端末の開発を行った.V字モデルに従い,要求分析,基本設計,詳細設計を行い,グループで役割分担をし開発を行った.システム全体の設計としては,UMLを用いて方針を固めた.実装においては,優先度の高い機能を実装し,それぞれの単体テスト,結合テスト,総合テストを通して動作を確認し,評価を行った.検証を行う際,問題があった場合はそれぞれの設計に戻り,再度検証を行った.検証を行ったシステムを評価軸に沿って評価し,今後の課題について考察した.本システムは小規模店舗,中規模店舗を対象とし,コストを抑えたシステムとして先発性がある.本システムの開発が進めば,小規模店舗,中規模店舗での人手不足問題が解消される.本研究により本システムは今後拡張性があり,低コストで運用ができる可能性を持つシステムであることを確認した.