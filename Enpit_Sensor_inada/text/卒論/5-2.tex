%第5章:考察
%考察(起こったことに対する理由付けや解釈)を書く.主観的でよい.


\section{考察}

本システムにおいては,無線通信モジュールを用いた環境値収集デバイスの作成を行ったが,IEEE802.15.4はセンシング機器において有用であると考えられる.
今回用いたような小型かつ低消費電力で動作するモジュールが出回っており,身の回りの様々な場所に容易に設置可能となりうる.
これによって,大きな工場等だけではなく,家庭内など,小さな環境であってもセンサなどのデバイスが設置しやすくなり,一般用途でのセンシングも行いやすくなると考えられる.
一般用途での使用ができることで,これまで満たされなかったニーズにも対応可能となると考えられ,幅広く用いることができると考えられる.
しかしながら,今回開発を行った二酸化炭素濃度の計測においては,そのセンサの小型化,省電力化が進んでおらず,一般普及には時間がかかるのではないかと考えられる.
今回のセンサデバイスの消費電力の半分程度を二酸化炭素濃度センサが占め,また温湿度センサに比べると非常に大きな専有面積を必要とする.
以上のような理由で,二酸化炭素濃度センサは今の状況では一般で広く用いられるためにはセンサ自体の改善,および発展が必要となるのではないかと考えられる.
