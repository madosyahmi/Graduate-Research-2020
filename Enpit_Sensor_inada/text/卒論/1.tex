%まえがき
%1-1要約:第一文-論文の趣旨,後1~2文①考えたこと②やったこと③結果の概要④成果の意義
%1-2問題設定(序論):研究をなぜ実施しなければならないのかを書く
%①あなたが解こうとする問題は何ですか?②その問題には必要性や需要はありますか?③あなたの取り組む問題は未解明で新規のものですか?④有用性はどれくらい見込めますか?
%1-3関連研究を引用.1)定石とすべき先行研究 2)定石化されていない部分の分岐状況
%各質問に対する回答パターンについては本参照




近年,新型コロナウイルスなどの感染症の拡大が世界中で話題となっており,多くの場所に影響を及ぼしている.
例えば,あるアンケートによると全体の81.1\%の人が健康に関する意識が変化したと回答している\cite{ishiki}.
特に感染症に感染するリスクを気にする人も多く,リスクに関しての情報を正しく得るということが求められる.
新型コロナウイルスを例にとってみると,飛沫感染,および接触感染によって感染すると言われている.
特に,密閉,密集,密接といった,いわゆる「3密」によって感染リスクが高くなると言われている\cite{koronaQA}.

情報が知れ渡るとともに,次はそのような状況を作らないことが同時に求められるようになっており,各方面から呼びかけが行われている.
それらの状況を作り出さないための方法として,例えば換気が挙げられるが,実際に換気によってどの程度空気環境が変わったのかは目で見てわかるものではない.
換気によって変わった空気環境を測るための基準としては浮遊粉塵の量,一酸化炭素濃度,二酸化炭素濃度,温度,湿度などが挙げられている\cite{kanki}が,これらすべてを計測し,判断することは容易なことではない.
この中でも温度,湿度については安価に計測機器を手に入れることができるが,それだけでは感染リスクという面から見た空気状況の判断が十分に行えるとは言い難い.
一方で,愛媛大学工学部社会基盤iセンシングセンターの実験によれば,部屋の換気状況の指標として二酸化炭素濃度の計測が有用であると思われるとの結果が出ており,この計測が重要となる.

しかしながら,二酸化炭素濃度も計測できる機器においてはコンセントから常時電源供給が必要であるものがほとんどとなっている.
これではコンセントなどの電源が供給するものが近くに必要となるなど,設置場所が限られてしまう.
また,それによって複数箇所に設置するのが難しくなるため,部屋の一か所の状況しか知ることができない場合が多い.
それでは,部屋全体の状況ではなく設置する場所の特性を反映したものになってしまい,正確なデータが収集しにくいという面を持つ.
このように,設置場所が限られてしまうなどにより,理想的な場所や複数箇所に設置できないため,安易に導入できないという大きな障害が生じている.

そこで,感染リスクを環境値だけでなく,その値の総合的な良し悪しを示すことができ,かつセンサの設置場所の制限が従来のものより少ないものを作成することが必要であると考えた.
これにより,感染症への意識の高まりからくる需要を満たすことができ,それとともに感染症が広がりにくい状況を構築する一助となることが期待できる.

そこで,本研究では,部屋の中の環境を総合的な観点からモニタリングし,その感染症リスクを分かりやすく表示でき,かつそれが容易に設置できる感染症予防サポートシステムの作成を目的とした.
この目的を達成するために,本研究では,乾電池で動作し,かつ無線でセンサのデータを送信する,従来より設置場所の制約の少ない小型の室内環境値計測デバイスを開発することを目標とする.

感染症予防サポートシステム全体の開発においては,システムの設計をより洗練されたものとし,かつ短期間で行うことができるよう,グループ(伊藤大輝,稲田一輝,小田恵吏奈,掛水誠矢)で開発を行った.
開発工程においてはグループ内での分担,および設計に対する検証を容易に行うためにV字開発モデルに従って行った.
また,要求分析,基本設計,詳細設計においてはグループ内での共通認識を図るためUML(Unified Modeling Language)を使用した.

本論文の構成は下記のとおりである.
%ここから下は後で修正%
第2章では本研究で用いる用語や研究方針,本システム全体の概要について述べる.
第3章ではV字開発モデルに従った本システムの設計について述べる.
第4章では,感染症予防サポートシステムとしての環境値取集デバイスの実装と検証結果について述べる.
第5章では実装・検証した本システムの評価を行い,考察を示す.
第6章では本研究のまとめを述べる.

%ここから去年
%しかしながら,売り場規模別のセルフレジの設置率については,大規模店舗中心型が25\%を越えているのに対し,小規模や中規模の企業はそれぞれ7.1\%,7.4\%\cite{super}と低い状態となっている.
%また,今後のセルフレジの設置意向について,全国スーパーマーケット協会によるアンケートに,セルフレジを新たに設置したいと回答した割合が,都市圏では8.8\%なのに対し,地方圏では14.0\%\cite{super}と高くなっている.
%人手不足が続くなか,利用者のレジ待ち時間を解消するため,精算スピードが速くなるセルフ精算レジの導入意向が高くなっていることが分かる\cite{super}.
%人手不足の著しい地方圏のスーパーマーケットや小規模や中規模の企業へセルフレジの導入が進んでいない理由としては,コストがかかることが要因として挙げられる.
%無人レジ店舗においては数十台のカメラやセンサが必要であったり,商品すべてに独自のICタグを埋め込む必要があったりなど大きなコストを要するものとなっている.
%また,既存のスーパーマーケットにおいても,セルフレジの導入は費用の点で大きな負担がかかっているのが現実の問題としてあることが考えらえる.
%
%そこで,本研究では既存の無人レジ店舗のような複雑で高価なシステムではなく,小規模や中規模の企業でも導入できる安価なスマートモビリティレジシステムの作成を本研究の目的とした.
%この目標を達成するために,本研究では,Webカメラと超音波センサ,ロードセルなどのセンサを用い,安価なモビリティショッピング端末を開発することを目標とする.
%
%具体的には,シングルボードコンピュータであるRaspberry PiとWebカメラ,各種センサを用い,商品の識別から決済に至るまでの一連の流れを行えるシステムの開発を行った.
%システムの開発では,V字モデルに従って,グループ(段原丞治,真鍋樹)でスマートモビリティレジシステムの開発を行った.要求分析,基本設計,詳細設計の際はUML(Unified Modeling Language)を用いた.
%
%本論文の構成は下記のとおりである.第2章では本研究で用いる用語や研究方針,本システム全体の概要について述べる.
%第3章ではV字モデルに従った本システムの設計について述べる.第4章では,モビリティショッピング端末の実装と検証結果について述べる.
%第5章では実装・検証した本システムの評価を行い,考察を示す.第6章では本研究のまとめを行う.